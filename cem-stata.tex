\documentclass[11pt,titlepage]{article}
%\usepackage[notref]{showkeys}
\usepackage[reqno]{amsmath}
\usepackage{natbib}
\usepackage{amssymb}
\usepackage{graphicx}
\usepackage{hyperref}
\usepackage[all]{xy}
\usepackage{verbatim}
%\usepackage{listings}
\usepackage{vmargin}
\setpapersize{USletter}
\topmargin=0in
\usepackage{url}
\usepackage[usenames]{color}
\usepackage{dcolumn}
\newcolumntype{.}{D{.}{.}{-1}}
\newcolumntype{d}[1]{D{.}{.}{#1}}
\newcommand{\vs}{\vspace{-\baselineskip}}
\newcommand{\obs}{{\text{obs}}}
\newcommand{\mis}{{\text{mis}}}
\newcommand{\cem}{\texttt{cem }}


\title{\tt{cem}\sf{: Coarsened Exact Matching in Stata}} \author{Matthew
  Blackwell\thanks{Institute for Quantitative Social Science,1737
    Cambridge Street, Harvard University, Cambridge MA 02138;
    \texttt{blackwel@fas.harvard.edu}).}
%
\and %
%
Stefano Iacus\thanks{Department of Economics, Business and Statistics,
  University of Milan, Via Conservatorio 7, I-20124 Milan, Italy; 
  \texttt{stefano.iacus@unimi.it}.}
%
\and %
%
Gary King\thanks{Institute for Quantitative Social
  Science, 1737 Cambridge Street, Harvard University, Cambridge MA
  02138; \texttt{http://GKing.Harvard.Edu}, \texttt{King@Harvard.Edu},
  (617) 495-2027).}
%
\and %
%
Giuseppe Porro\thanks{Department of Economics and Statistics, University of
  Trieste, P.le Europa 1, I-34127 Trieste, Italy; 
  \texttt{giuseppe.porro@econ.units.it}.}}
\title{cem: Coarsened Exact Matching in Stata} 

\begin{document}
\maketitle

\begin{abstract}
  This paper introduces a Stata implementation of Coarsened Exact
  Matching (CEM), a new method for improving the estimation of causal
  effects by reducing imbalance in covariates between treated and
  control groups.  CEM is faster, easier to use and understand,
  requires fewer assumptions, more easily automated, and possesses
  more attractive statistical properties for many applications than
  existing matching methods.  In CEM, users temporarily coarsen their
  data, exact match on these coarsened data, then run their analysis
  on the uncoarsened, matched data.  CEM bounds the degree of model
  dependence and causal effect estimation error by ex ante user
  choice, is montonic imbalance bounding (so that reducing the maximum
  imbalance on one variable has no effect on others), does not require
  a separate procedure to restrict data to common support, meets the
  congruence principle, is approximately invariant to measurement
  error, balances all nonlinearities and interactions in-sample (i.e.,
  not merely in expectation), and works with multiply imputed data
  sets.  Other matching methods inheret many of CEM's properties when
  applied to further match data preprocessed by CEM.  The library
  \texttt{cem} implements the CEM algorithm in Stata.
\end{abstract}


\section{Introduction}

This program is designed to improve the estimation of causal effects
via a powerful method of matching that is widely applicable in
observational data and easy to understand and use (if you understand
how to draw a histogram, you will understand this method).  The
program implements the Coarsened Exact Matching (CEM) algorithm
described in \citep{IacKinPor12}.  CEM is a monotonoic imbalance
reducing matching method --- which means that the balance between the
treated and control groups is chosen by ex ante user choice rather
than discovered through the usual laborious process of checking after
the fact, tweaking the method, and repeatedly reestimating.  CEM also
assures that adjusting the imbalance on one variable has no effect on
the maximum imbalance of any other.  CEM strictly bounds through ex
ante user choice both the degree of model dependence and the average
treatment effect estimation error, eliminates the need for a separate
procedure to restrict data to common empirical support, meets the
congruence principle, is robust to measurement error, works well with
multiple imputation methods for missing data, can be completely
automated, and is extremely fast computationally even with very large
data sets.  After preprocessing data with CEM, the analyst may then
use a simple difference in means or whatever statistical model they
would have applied without matching.  CEM can also be used to improve
other methods of matching by applying those methods to CEM-matched
data (they formally inherent CEM's properties if applied within CEM
strata).  CEM also works well for determining blocks in randomized
experiments, and evaluating extreme counterfactuals.

\section{Background}

\subsection{Notation}

Consider a sample of $n$ units  randomly drawn from a population of $N$
units, where $n \leq N$.  For unit $i$, denote $T_i$ as an indicator
variable with value $T_i=1$ if unit $i$ receives the treatment (and so
is a member of the ``treated'' group) and $T_i=0$ if not (and is
therefore a member of the ``control'' group).  The outcome variable is
denoted $Y$, where $Y_i(0)$ is the potential outcome for observation
$i$ if the unit does not receive treatment and $Y_i(1)$ is the
potential outcome if the (same) unit receives treatment.  For each
observed unit, the observed outcome is $Y_i = T_i Y_i(1) + (1-T_i)
Y_i(0)$ and so $Y_i(0)$ is unobserved if $i$ receives treatment and
$Y_i(1)$ is unobserved if $i$ does not receive treatment.

To compensate for the observational data problem where the treated and
control groups are not necessarily identical before treatment (and,
lacking random assignment, not the same on average), matching
estimators attempt to control for pre-treatment covariates.  For this
purpose, we denote $X=(X_1, X_2, \ldots, X_k)$ as a $k$-dimensional
data set, where each $X_j$ is a column vector of  observed
values of pre-treatment variable $j$ for the $n$ sample observations
(possibly drawn from a population, of size $N$).  That is, $X =
[X_{ij}, i=1, \ldots, n, j=1, \ldots, k]$.
%$$
%X = [X_{ij}, i=1, \ldots, n, j=1, \ldots,
%k]= \left(\begin{array}{cccc}
%    X_{11} & X_{12} & \cdots & X_{1k} \\
%    X_{21} & X_{22} & \cdots & X_{2k} \\
%    \vdots & \vdots & \dots & \vdots \\
%    X_{n1} & X_{n2} & \cdots & X_{nk}
%\end{array}\right)
%$$

\subsection{Quantities of Interest}

As usual, the treatment effect for unit $i$, ${\rm TE}_i = Y_i(1) -
Y_i(0)$, is unobserved.  All relevant causal quantities of interest
are functions of $\textrm{TE}_i$, for different groups of units, and
so must be estimated.  We focus on the sample average treatment effect
on the treated (SATT):
\begin{equation} 
  {\rm SATT} = \frac{1}{n_T} \sum_{i\in T} {\rm TE}_i
\end{equation}
where $n_T = \sum_{i=1}^n T_i$ and $T = \{1 \leq i \leq n : T_i=1\}$.
Matching algorithms sometimes also change the quantity being estimated
to one that can be estimated without much model dependence by
selecting control and/or treated units.

We assume that treatment assignment is ignorable conditional on $X$. This
assumption is often stated as ``no unmeasured confounders'' or ``no
omitted variables.'' Formally, this means that the treatment assignment is
independent of the potential outcomes, 
\begin{equation}
P(T|X,Y(0),Y(1)) = P(T|X).
\end{equation}

\subsection{Existing matching methods and practice}

Matching is a nonparametric method of controlling for some or all of
the confounding influence of pretreatment control variables in
observational data.  The key goal of matching is to prune observations
from the data so that the remaining data have better \emph{balance}
between the treated and control groups, meaning that the empirical
distributions of the covariates ($X$) in the groups are more similar.
Exactly balanced data means that controlling further for $X$ is
unnecessary (since it is unrelated to the treatment variable), and so
a simple difference in means on the matched data can estimate the
causal effect; approximately balanced data requires controlling for
$X$ with a model (such as the same model that would have been used
without matching), but the only inferences necessary are only those
relatively close to the data, leading to less model dependence and
reduced statistical bias than without matching.

The most common matching methods involve finding, for each treated unit,
at least one control unit that is ``similar'' on the covariates.  The
distinction between methods is how to define this similarity. For example,
\emph{exact matching} simply matches a treated unit to all of the control
units with the same covariate values. Unfortunately, due to the richness
of covariates in many examples, this method often produces very few
matches. A whole host of \emph{approximate matching} methods specify a
metric to find control units that are close to the treated unit. This
metric is often the Mahalanobis distance or the propensity score (which is
simply the probability of being treated, conditional on the covariates).
Many of these related methods are implemented in Stata  \citep{SasIch02,
  AbaDruHer01, LeuSia04, AbaDiaHai09}.  A problem with this type of solution is
that it requires the user to set the size of the matching solution ex
ante, then check for balance ex post. Thus analysts must check for balance
after the algorithm is finished, then respecify a matching model and
recheck balance, etc. This process repeats until the user obtains an
acceptable amount of balance.

As matching is simply a data preprocessing technique, analysts must still
apply statistical estimators to the data after matching. When one-to-one
exact matching is used, a simple difference in means between $Y$ in the
treated and control group provides an estimator of the causal effect. When
the match is not exact, a parametric model must be used to control for the
differences in the covariates across treated and control groups. This
may be a linear regression, a maximum likelihood estimator or some other
estimator. Applying a matching method to the data before analysis can
reduce the degree of model dependence \citep{HoImaKin07}. 

One wrinkle in the analysis of matched data occurs when there are not
equal numbers of treated and control units within strata. In this
situation, estimators require weighting observations according to the size
of their strata \citep{IacKinPor12}.

\section{Coarsened Exact Matching} 

\subsection{The Algorithm}

The central motivation for CEM is that while exact matching provides
perfect balance, it typically produces few matches due to
curse-of-dimensionality issues. For instance, adding one continuous
variable to a dataset effectively kills exact matching since two
observations are unlikely to have identical values on a continuous
measure. The idea of CEM is to temporarily coarsen each variable into
substantively meaningful groups, exact match on these coarsened data
and then only retain the original (uncoarsened) values of the matched
data.  As coarsening is a process at the heart of measurement, many
analysts know how to coarsen a variable into groups that preserve
information. For instance, education may be measured in years, but
many would be comfortable grouping observations into categories of
high school, some college, college graduates, etc. This method works
by exact matching on distilled information in the covariates as chosen
by the user.

The algorithm works as follows:
\begin{enumerate}
\item Begin with the covariates $X$ and make a copy, which we denote
  $X^*$.
\item Coarsen $X^*$ according to user-defined cutpoints, or CEM's
  automatic binning algorithm.
\item Create one stratum per unique observation of $X^*$ and place each
  observation in a stratum.
\item Assign these strata to the original data, $X$ and drop any
  observation whose stratum does not contain at least one treated and
  one control unit.
\end{enumerate}

Once completed, these strata are the foundations for calculating the
treatment effect.  The inherent trade-off of matching is reflected in
CEM too: larger bins (more coarsening) used to make $X^*$ will result
in fewer strata. Fewer strata will result in more diverse observations
within the same strata and, thus, higher imbalance.

It is important to note that CEM prunes both treated and control
units.  This process changes the quantity of interest under study to
the treatment effect in the post-matching subsample. This change is
reasonable so long as the decision is transparent (see e.g.
\citet{CruHotImb06}).

\subsection{The Benefits}
\citet{IacKinPor12} derive many of the properties of the CEM algorithm and
we review some of them here. The key property of CEM is that it is in a
class of matching methods called Monotonic Imbalance Bounding (MIB). MIB
methods bound the maximum imbalance in some feature of the empirical
distributions through an \emph{ex ante} choice by the user. In CEM, this
ex ante choice is the coarsening. As the coarsening on any variable
becomes finer (the bins become more narrow), the bound on the maximum
imbalance on the moments of that variable becomes tighter. This is also
true for the bound on differences in the empirical quantiles.
Furthermore, this choice also bounds the maximum imbalance on the full
multivariate histogram of treated and control units, which includes all
interactions and non-linearities. By choosing the coarsening ex ante,
users can control the amount of imbalance in the matching solution.
\citet{IacKinPor12} also show that CEM bounds both the error in estimating the
average treatment effect and the amount of model dependence.

Aside from bounding the imbalance between the treated and control groups,
CEM has a number of other beneficial properties. First, CEM meets the
\emph{congruence principle}, which states that the data space and analysis
space should be the same. Methods that fail to meet this principle often
produce strange or counter-intuitive results. Methods that meet the
principle allow analysts to leverage their substantive knowledge of the
data in order to find better matches. Second, CEM automatically restricts
the matched data to areas of common empirical support. This is necessary to
remove the possibility of difficult-to-justify extrapolations of the
causal effect that end up being heavily model dependent
\citep{KinZen06}. Finally, CEM is computationally very efficient even
for large data sets.

\section{An Extended Example}
We show here how to use CEM\footnote{In addition to the Stata version
  of CEM, there is an R version in the package \texttt{cem}. The
  example presented here is also used in that package as a vignette,
  and includes some obvious overlap in prose.} through a simple
running example: the National Supported Work (NSW) Demonstration data,
also known as the Lalonde data set \citep{Lalonde86}.  This program
provided training to selected individuals for 12-18 months and help
finding a job in the hopes of increasing their earnings.  The
treatment variable, \texttt{treated}, is 1 for participants (the
treatment group) and 0 for nonparticipants (the control group).  The
key outcome variable is earnings in 1978 (\texttt{re78}).  The
statistical goal is to estimate a specific version of a causal effect:
the sample average treatment effect on the treated (the ``SATT'').

Since participation in the program was not assigned strictly at random, we
must control for a set of pretreatment variables by the CEM algorithm.
These pre-treatment variables include age (\texttt{age}), years of
education (\texttt{education}), marital status (\texttt{married}), lack of
a high school diploma (\texttt{nodegree}), race (\texttt{black},
\texttt{hispanic}), indicator variables for unemployment in 1974
(\texttt{u74}) and 1975 (\texttt{u75}), and real earnings in 1974
(\texttt{re74}) and 1975 (\texttt{re75}).  Some of these are dichotomous
(\texttt{married}, \texttt{nodegree}, \texttt{black}, \texttt{hispanic},
\texttt{u74}, \texttt{u75}), some are categorical (\texttt{age} and
\texttt{education}), and the earnings variables are continuous and highly
skewed with point masses at zero. You can load this data into Stata using
the command

\begin{verbatim}
use http://gking.harvard.edu/cem/lalonde.dta, clear
\end{verbatim}

Matching is not a method of estimation; it is a way to preprocess a
data set so that estimation of SATT based on the matched data set will
be less ``model-dependent'' (i.e., less a function of apparently small
and indefensible modeling decisions) than when based on the original
full data set.  Matching involves pruning observations that have no
close matches on pre-treatment covariates in both the treated and
control groups.  The result is typically less model-dependence, lower bias,
and (by removing heterogeneity) increased efficiency
\citep{KinZen06,HoImaKin07,IacKinPor12}.

\subsection{Basic Evaluation and Analysis of Unmatched Data}\label{s:basic}

We begin the simple difference in means as a naive estimate of SATT;
this estimator is useful only when the in-sample distribution of
pre-treatment covariates happens to be the same in the treatment and
control groups. First we compute the size of the treated and control
groups:

\begin{verbatim}
. table treated

----------------------
  treated |      Freq.
----------+-----------
        0 |        425
        1 |        297
----------------------

\end{verbatim}


Thus, the data include 297 treated units and 425 control units.  The
(unadjusted and therefore likely biased) difference in means can be found
by a simple linear regression of outcome on treatment,

\begin{verbatim}

. regress re78 treated

      Source |       SS       df       MS              Number of obs =     722
-------------+------------------------------           F(  1,   720) =    3.52
       Model |   137332528     1   137332528           Prob > F      =  0.0609
    Residual |  2.8053e+10   720  38962865.4           R-squared     =  0.0049
-------------+------------------------------           Adj R-squared =  0.0035
       Total |  2.8191e+10   721  39099300.5           Root MSE      =    6242

------------------------------------------------------------------------------
        re78 |      Coef.   Std. Err.      t    P>|t|     [95% Conf. Interval]
-------------+----------------------------------------------------------------
     treated |   886.3038   472.0863     1.88   0.061    -40.52625    1813.134
       _cons |   5090.048   302.7826    16.81   0.000     4495.606    5684.491
------------------------------------------------------------------------------

\end{verbatim}


Thus, our estimate of SATT is 886.3. Because the variable \texttt{treated} was
not randomly assigned, the pre-treatment covariates differ between the
treated and control groups.  To see this, we focus on these pre-treatment
covariates: \texttt{age}, \texttt{education}, \texttt{black},
\texttt{nodegree}, \texttt{re74}.

The overall imbalance is given by the $\mathcal L_1$ statistic,
introduced in \citet{IacKinPor12} as a comprehensive measure of global
imbalance.  It is based on the $L_1$ difference between the
multidimensional histogram of all pretreatment covariates in the
treated group and that in the control group.  First, we coarsen the
covariates into bins. To use this measure, we require a list of bin
sizes for the numerical variables.  Our functions compute these
automatically, or they can be set by the user.\footnote{Of course, as
  with drawing histograms, the choice of bins affects the final
  result.  The crucial point is to choose one and keep it the same
  throughout to allow for fair comparisons.  The particular choice is
  less crucial.}  Then, we cross-tabulate the discretized variables as
$X_1\times \dots \times X_k$ for the treated and control groups
separately, and record the $k$-dimensional relative frequencies for
the treated $f_{\ell_1\cdots \ell_k}$ and control $g_{\ell_1\cdots
  \ell_k}$ units.  Finally, our measure of imbalance is the absolute
difference over all the cell values:
\begin{equation}
  \mathcal L_1(f,g) = \frac12 \sum_{\ell_1  \cdots \ell_k} 
  |f_{\ell_1\cdots \ell_k} - g_{\ell_1\cdots \ell_k}|
  \label{eq:L1}
\end{equation}

Perfect global balance (up to coarsening) is indicated by $\mathcal
L_1=0$, and larger values indicate larger imbalance between the
groups, with a maximum of $\mathcal L_1=1$, which indicates complete
separation. If we denote the relative frequencies of a matched dataset
by $f^m$ and $g^m$, then a good matching solution would produce a
reduction in the $\mathcal L_1$ statistic; that is, we would hope to
have $\mathcal L_1(f^m, g^m) \leq \mathcal L_1(f,g)$.

We compute $\mathcal L_1$ statistic, as well as several unidimensional
measures of imbalance via our \texttt{imb} function.  In our
running example:

\begin{verbatim}
. imb age education black nodegree re74, treatment(treated)

Multivariate L1 distance: .50759358

Univariate imbalance:

                L1     mean      min      25%      50%      75%      max
      age   .10119    .1792        0        1        0       -1       -6
education   .10047   .19224        1        0        1        1        2
    black   .00135   .00135        0        0        0        0        0
 nodegree   .08348  -.08348        0       -1        0        0        0
     re74    .0522  -101.49        0        0   69.731   584.92    -2139

\end{verbatim}

Only the overall $\mathcal L_1$ statistic measure includes imbalance with
respect to the full joint distribution, including all interactions, of the
covariates; in the case of our example, $\mathcal L_1=.5076$. The
$\mathcal L_1$ value is not valuable on its own, but rather as a point of
comparison between matching solutions. The value .5076 is a baseline
reference for the unmatched data. Once we have a matching solution, we
will compare its $\mathcal L_1$ value to .5076 and gauge the increase in
balance due to the matching solution from that difference. Thus, $\mathcal
L_1$ works for imbalance as $R^2$ works for model fit: the absolute values
mean less than comparisons between matching solutions. The unidimensional
measures in the table are all computed for each variable separately.

The first column, labeled \texttt{L1}, reports the $\mathcal L_1^j$
measure, which is $\mathcal L_1$ computed for the $j$-th variable
separately (which of course does not include interactions).  The second
column in the table of unidimensional measures, labeled \texttt{mean},
reports the difference in means.  The remaining columns in the table
report the difference in the empirical quantiles of the distributions of
the two groups for the 0th (min), 25th, 50th, 75th, and 100th (max)
percentiles for each variable. 

This particular table shows that variables \texttt{re74} is imbalanced in
the raw data in many ways and variable \texttt{age} is balanced in means
but not in the quantiles of the two distributions.  This table also
illustrates the point that balancing only the means between the treated
and control groups does not necessarily guarantee balance in the rest of
the distribution.  Most important, of course, is the overall $\mathcal
L_1$ measure, since even if the marginal distribution of every variable is
perfectly balanced, the joint distribution can still be highly imbalanced.

\subsection{Coarsened Exact Matching}\label{sec:cem}

We now apply the coarsened exact matching algorithm by calling the
function \texttt{cem}.  The CEM algorithm performs exact matching on
coarsened data to determine matches and then passes on the uncoarsened
data from observations that were matched to estimate the causal
effect.  Exact matching works by first sorting all the observations
into strata, each of which has identical values for all the coarsened
pre-treatment covariates, and then discarding all observations within
any stratum that does not have at least one observation for each
unique value of the treatment variable.

To run this algorithm, we must choose a type of coarsening for each
covariate.  We show how this is done this via a fully automated procedures
in next section.  Then we show how to use explicit prior knowledge to
choose the coarsening, which is normally preferable when feasible.

In CEM, the treatment variable may be \emph{dichotomous} or
\emph{multichotomous}\footnote{ While CEM can match for multichotomous
  treatments, analysis with these matched samples is somewhat difficult.
  For instance, \citet{IacKinPor12} develop weights for two treatment
  groups and it is not obvious how to generalize these weights for more
  treatment groups. We suggest users run CEM on each pair of treatment
  levels, get the correct weights for each and calculate separate ATT.} .
Alternatively, \texttt{cem} may be used for \emph{randomized block
  experiments} without specifying a treatment variable; in this case the
strata are simply returned without any pruning of observations.

\subsubsection{Automated Coarsening}\label{s:cem-auto}

In our running example we have a dichotomous treatment variable.  In the
following code, we match on our chosen pre-treatment variables, but not
\texttt{re78}, which is the outcome variable and so should never be
included.

The output contains useful information about the match, including a
(small) table about the number of observations in total, matched, and
unmatched by treatment group, as well as the results of a call to the
\texttt{imb} function for information about the quality of the
matched data. Since \texttt{cem} bounds the imbalance ex ante, the most
important information is the number of observations matched.  But the
results also give the imbalance in the matched data using the same
measures as that in the original data described in Section \ref{s:basic}.
Thus,

\begin{verbatim}
. cem age education black nodegree re74, tr(treated)

Matching Summary:
-----------------
Number of strata: 205
Number of matched strata: 67

             0    1
      All  425  297
  Matched  324  228
Unmatched  101   69


Multivariate L1 distance: .46113967

Univariate imbalance:

                 L1      mean       min       25%       50%       75%
      age    .13641   -.17634         0         0         0         0
education    .00687    .00687         1         0         0         0
    black   3.2e-16  -2.2e-16         0         0         0         0
 nodegree   5.8e-16   4.4e-16         0         0         0         0
     re74    .06787    34.438         0         0    492.23    39.425

                max
      age        -1
education         0
    black         0
 nodegree         0
     re74    96.881

\end{verbatim}


We can see from these results the number of observations matched and
thus retained, as well as those which were pruned because they were
not comparable.  By comparing the imbalance results to the original
imbalance table given in the previous section, we can see that a good
match can produce a substantial reduction in imbalance, not only in
the means, but also in the marginal and joint distributions of the
data.

The function \texttt{cem} also generates weights for use in the
evaluation of imbalance measures and estimates of the causal effect
(stored in \texttt{cem\_weights}).

\subsubsection{Coarsening by Explicit User Choice}\label{s:cem-user}

The power and simplicity of CEM comes from choosing the coarsening
yourself rather than using the automated algorithm as in the previous
section.  Choosing the coarsening enables you to set the maximum level
of imbalance ex ante, which is a direct function of the coarsening you
choose.  By controlling the coarsening, you also put an explicit bound
on the degree of model dependence and the SATT estimation error.

Fortunately, the coarsening is a fundamentally substantive act, almost
synonymous with the measurement of the original variables.  In other
words, if you know something about the data you are analyzing, you
almost surely have enough information to choose the coarsening.  (And
if you don't know something about the data, you might ask why you are
analyzing it in the first place!) 

In general, we want to set the coarsening for each variable so that
substantively indistinguishable values are grouped and assigned the
same numerical value.  Groups may be of different sizes if
appropriate.  Recall that any coarsening during CEM is used only for
matching; the original values of the variables are passed on to the
analysis stage for all matched observations.

For numerical variables, we can use the cutpoints syntax in \texttt{cem}.
Thus, for example, in the US educational system, the following
discretization of years of education corresponds to different levels of
school
\begin{center}
\begin{tabular}{ll}
Grade school    & 0--6\\
Middle school   & 7--8\\
High school     & 9--12\\
College         & 13--16\\
Graduate school & $>$16 
\end{tabular}
\end{center}
Using these natural breaks in the data to create the coarsening is
generally a good approach and certainly better than using fixed bin sizes
(as in caliper matching) that disregard these meaningful breaks.  In our
data, no respondents fall in the last category,

\begin{verbatim}
. table education

----------------------
education |      Freq.
----------+-----------
        3 |          1
        4 |          6
        5 |          5
        6 |          7
        7 |         15
        8 |         62
        9 |        110
       10 |        162
       11 |        195
       12 |        122
       13 |         23
       14 |         11
       15 |          2
       16 |          1
----------------------

\end{verbatim}

We can use the cutpoints above using parentheses after the
\texttt{education} variable:

\begin{verbatim}
. cem age education (0 6.5 8.5 12.5 17.5) black nodegree re74, tr(treated)

Matching Summary:
-----------------
Number of strata: 155
Number of matched strata: 53

             0    1
      All  425  297
  Matched  349  245
Unmatched   76   52


Multivariate L1 distance: .43604654

Univariate imbalance:

                L1     mean      min      25%      50%      75%      max
      age   .05034  -.15556        0        0        0        1       -1
education    .0309   .00362        1       -1        0        0        2
    black  8.2e-16  1.0e-15        0        0        0        0        0
 nodegree  1.2e-15  1.9e-15        0        0        0        0        0
     re74   .04975   2.5048        0        0   161.28   -17.37   1198.1

\end{verbatim}

As we can see, this matching solution differs from that resulting from our
automated approach in the previous section. In fact, it has actually
increased the balance in matching solution while giving us a higher number
of matched units.

\subsubsection{Coarsening categorical variables}

For categorical variables that do not have a natural ordering, some
recoding might be necessary before inputing to CEM. For instance, if we
have a variable that is

\begin{center}
\begin{tabular}{ll}
Strongly Agree    & 1\\
Agree   & 2\\
Neutral     & 3\\
Disagree         & 4\\
Strongly Disagree & 5\\
No Opinion      & 6 
\end{tabular}
\end{center}
there is a category (``No Opinion'') that does not fit on the ordinal
scale of the variable. In our example dataset, we have such a variable,
\texttt{q1},
\begin{verbatim}

. table q1

------------------------------
               q1 |      Freq.
------------------+-----------
   strongly agree |        121
            agree |        111
          neutral |        129
         disagree |        121
strongly disagree |        118
       no opinion |        122
------------------------------
\end{verbatim}

In order to coarsen this variable, first create a new coarsened variable
using the \texttt{recode} command\footnote{For variables that are strictly
  string (non-numeric) variables, users will need to first use the
  \texttt{encode} command to convert the strings to numeric, then use
  \texttt{recode}.}:
\begin{verbatim}

. recode q1 (1 2 = 1 "agree") (3 6 = 2 "neutral") (4 5 = 3 "disagree"), gen(cem_q1)
(601 differences between q1 and cem_q1)

. table cem_q1

----------------------
RECODE of |
q1        |      Freq.
----------+-----------
    agree |        232
  neutral |        251
 disagree |        239
----------------------

\end{verbatim}

Here we have collapsed the opinions into the direction of opinion, also
grouping ``No Opinion'' with ``Neutral.''  Once the coarsened variable is
created, you can pass this variable to CEM with the \texttt{(\#0)}
cutpoints command after it to ensure that CEM does not coarsen further:

\begin{verbatim}

. cem age education black nodegree re74 cem_q1 (#0), tr(treated)

Matching Summary:
-----------------
Number of strata: 315
Number of matched strata: 81

             0    1
      All  425  297
  Matched  260  190
Unmatched  165  107


Multivariate L1 distance: .5904067

Univariate imbalance:

                L1     mean      min      25%      50%      75%      max
      age   .14574   -.1994        0        0        0        1       -1
education   .00263   .00263        1        0        0        0        0
    black  3.6e-16  6.7e-16        0        0        0        0        0
 nodegree  3.5e-16  6.7e-16        0        0        0        0        0
     re74   .09854   70.061        0        0    375.1  -383.76   96.881
   cem_q1  3.1e-16  3.1e-15        0        0        0        0        0

\end{verbatim}

When calculating treatment effects after running CEM, be sure to use the
original, uncoarsened variables for analysis. Coarsened variable should
only be used to produce matches. After this, they can be discarded. 


\subsection{Restricting the matching solution to a $k$-to-$k$ match}\label{s:k2k}

By default, CEM uses maximal information, resulting in strata that may
include different numbers of treated and control units.  To compensate for
the differential strata sizes, \texttt{cem} also returns weights to be
used in subsequent analyses.  Although this is generally the best option,
a user with enough data may opt for a $k$-to-$k$ solution to avoid the
slight inconvenience of needing to use weights.

The argument \texttt{k2k} accomplishes this by pruning observations from a
\texttt{cem} solution within each stratum until the solution contains the
same number of treated and control units within all strata.  Pruning
occurs within a stratum (for which observations are indistinguishable to
\cem proper) by random matching inside \texttt{cem} strata\footnote{Note
  that in the R version of this software pruning within strata can be done
  using a distance metric.}.

Here is an example of this approach. Running the earlier call with the
\texttt{k2k} options yields:

\begin{verbatim}
. cem age education black nodegree re74, tr(treated) k2k

Matching Summary:
-----------------
Number of strata: 205
Number of matched strata: 67

             0    1
      All  425  297
  Matched  205  205
Unmatched  220   92


Multivariate L1 distance: .37560976

Univariate imbalance:

                L1     mean      min      25%      50%      75%      max
      age   .07805  -.10732        0        0        0        0       -1
education        0        0        1        0        0        0        0
    black        0        0        0        0        0        0        0
 nodegree        0        0        0        0        0        0        0
     re74    .0439  -34.547        0        0   -120.7  -214.55   96.881

\end{verbatim}

It is clear that the number of matched units has decreased after using the
\texttt{k2k} option. 

\subsection{Estimating the Causal Effect from $\tt cem$ output}

Using the output from \texttt{cem}, we can estimate the SATT by the
regular Stata methods, by simply including the \texttt{cem\_weights}. For
example, 

\begin{verbatim}
. reg re78 treated [iweight=cem_weights]

      Source |       SS       df       MS              Number of obs =     552
-------------+------------------------------           F(  1,   550) =    3.15
       Model |   128314324     1   128314324           Prob > F      =  0.0766
    Residual |  2.2420e+10   550  40764521.6           R-squared     =  0.0057
-------------+------------------------------           Adj R-squared =  0.0039
       Total |  2.2549e+10   551  40923414.2           Root MSE      =  6384.7

------------------------------------------------------------------------------
        re78 |      Coef.   Std. Err.      t    P>|t|     [95% Conf. Interval]
-------------+----------------------------------------------------------------
     treated |   979.1905   551.9132     1.77   0.077    -104.9252    2063.306
       _cons |    4919.49   354.7061    13.87   0.000     4222.745    5616.234
------------------------------------------------------------------------------

\end{verbatim}

For convenience, we compute this as a regression of the outcome variable
on a constant and the treatment variable, where the SATT estimate is the
coefficient on the \texttt{treated} variable, in our case 979.19.  Any
Stata command that accepts weights (\texttt{aweight} or
\texttt{iweight}) can be used.

If exact matching (i.e., without coarsening) was chosen this procedure is
appropriate as is.  In other situations, with some coarsening, some
imbalance remains in the matched data.  The remaining imbalance is
strictly bounded by the level of coarsening, which can be seen by any
remaining variation within the coarsened bins.  Thus, a reasonable
approach in this common situation is to attempt to adjust for the
remaining imbalance via a statistical model.  (Modeling assumptions for
models applied to the matched data are much less consequential than they
would otherwise be because CEM is known to strictly bound the level of
model dependence.)  To apply a statistical model to control for the
remaining imbalance, we simply add variables to the regression command.
For example:

\begin{verbatim}
. reg re78 treated re74 re75 [iweight=cem_weights]

      Source |       SS       df       MS              Number of obs =     552
-------------+------------------------------           F(  3,   548) =    5.42
       Model |   649651702     3   216550567           Prob > F      =  0.0011
    Residual |  2.1899e+10   548  39961951.7           R-squared     =  0.0288
-------------+------------------------------           Adj R-squared =  0.0235
       Total |  2.2549e+10   551  40923414.2           Root MSE      =  6321.5

------------------------------------------------------------------------------
        re78 |      Coef.   Std. Err.      t    P>|t|     [95% Conf. Interval]
-------------+----------------------------------------------------------------
     treated |    988.083   546.5395     1.81   0.071    -85.48584    2061.652
        re74 |  -.0174322   .1593346    -0.11   0.913    -.3304134    .2955491
        re75 |   .3190651   .1744905     1.83   0.068     -.023687    .6618172
       _cons |   4287.523   393.0883    10.91   0.000     3515.378    5059.667
------------------------------------------------------------------------------

\end{verbatim}


The user can also specify \texttt{glm} modeling in the case of binary,
count, or other noncontinuous outcome variables by utilizing their
commands in Stata (\texttt{logit},\texttt{ poisson}, etc) combined with the
\texttt{iweight} syntax.

\subsection{Matching and Missing Data}\label{s:mv}

Almost all previous methods of matching assume the absence of any missing
values.  In contrast, CEM offers two approaches to dealing with
missing values (item nonresponse).  In the first, where we treat missing
values as one of the values of the variables, is appropriate when
``\texttt{.}'' is a valid value that is not really missing (such as when
``no opinion'' really means no opinion).  The other is a special procedure
to allow for multiply imputed data in CEM.

\subsubsection{Matching on Missingness}\label{s:mvdirect}

If users leave missing values in the data, \cem will coarsen the variables
as normal, but use ``\texttt{.}'' as a separate category for each
variable. Thus, \cem will match on missingness. 

\subsubsection{Matching Multiply Imputed Data in a Single File}\label{s:mvmiflong}

Consider a data set to be matched, some of which is missing. One
approach to analyzing data with missing values is \emph{multiple
imputation}, which involves creating $m$ (usually about $m=5$) data
sets, each of which is the same as the original except that the
missing values have been imputed in each.  Uncertainty in the values
of the missing cells is represented by variation in the imputations
across the different imputed data sets \citep{KinHonJos01}.

When one uses the \texttt{mi} commands in Stata to impute missing
data, the program can return the data in a stacked or \texttt{flong}
format. The \texttt{cem} command can take advantage of this format to
easily match multiply imputed datasets. Note that the data must be in
the \texttt{flong} format: one dataset with $\tilde{n} = n + m \times
n$ observations, where $m$ is the number of imputed datasets and $n$
is the number of observations in the original dataset. The first $n$
rows of the data are the original dataset and the following $m \times
n$ rows are each of the imputed datasets. When using the \texttt{mi}
commands in Stata, you can ensure the data are in this format by using
\texttt{mi set flong} or \texttt{mi convert flong}.

Once you have your imputed data in \texttt{flong} format, you can pass
the data to \texttt{cem} along with the imputation variable and
\texttt{cem} will match the multiple imputations. The CEM algorithm
first runs on the entire stacked dataset of imputations, with possibly
different strata for each imputation. To combine strata across
imputation, CEM chooses the strata most often assigned to an
observation. This strata assignment is given to each of the imputed
datasets (that is, the \texttt{cem\_weights} variable is added to each
of the datasets).

To run this command simply add the \texttt{impvar} option:

\begin{verbatim}
. use "http://www.mattblackwell.org/files/stata/imp-lelonde.dta", clear
(Written by R.              )

. mi import flong, m(imp) id(mi_id)

. cem age education black nodegree re74, tr(treated) impvar(imp)
(using the scott break method for imbalance)

Matching Summary:
-----------------
Number of strata: 244
Number of matched strata: 85

             0    1
      All  425  297
  Matched  299  229
Unmatched  126   68


Multivariate L1 distance: .37560605

Univariate imbalance:

                L1     mean      min      25%      50%      75%      max
      age   .08093  -.05183        0        0        0        0       -1
education   .01276  -.00457        0        0        0        0        0
    black   .00218   .00063        0        0        0        0        0
 nodegree   .00437  -.00092        0        0        0        0        0
     re74   .05257   1.1624        0        0    104.2   232.05   1401.7
\end{verbatim}
Now that we have run \texttt{cem}, we can then use \texttt{mi
  estimate} to easily combine regressions from the imputed datasets to
estimate the SATT \citep{KinHonJos01}, being sure to weight by the
\texttt{cem\_weights}:
\begin{verbatim}
. mi estimate: reg re78 treated [iweight=cem_weights]

Multiple-imputation estimates                     Imputations     =          5
Linear regression                                 Number of obs   =        528
                                                  Average RVI     =     0.0353
                                                  Largest FMI     =     0.0308
                                                  Complete DF     =        526
DF adjustment:   Small sample                     DF:     min     =     455.20
                                                          avg     =     472.43
                                                          max     =     489.65
Model F test:       Equal FMI                     F(   1,  455.2) =       4.04
Within VCE type:          OLS                     Prob > F        =     0.0450

------------------------------------------------------------------------------
        re78 |      Coef.   Std. Err.      t    P>|t|     [95% Conf. Interval]
-------------+----------------------------------------------------------------
     treated |   1112.818   553.5378     2.01   0.045     25.01139    2200.624
       _cons |   4867.038   362.5507    13.42   0.000     4154.691    5579.385
------------------------------------------------------------------------------

\end{verbatim}

In addition to the built-in multiple imputation tools, there are
several user-written tools for combining multiple regression output
such as \texttt{miest} written by Ken Scheve\footnote{\texttt{miest}
  is available at
  \url{http://gking.harvard.edu/amelia/amelia1/docs/mi.zip}.} and
\texttt{clarify}\footnote{Clarify is available at
  \url{http://gking.harvard.edu/stats.shtml\#clarify}}.

\subsubsection{Matching Multiply Imputed Data in Separate Files}\label{s:mvmi}

Suppose that we have used some imputation program (such as
\citep{HonKinBLa10}) to produce 5 imputed datasets, saved as:

\begin{verbatim}
imp1.dta
imp2.dta
imp3.dta
imp4.dta
imp5.dta
\end{verbatim}

As an example we added missingness to the example dataset and imputed it
using \citet{HonKinBLa10}\footnote{If users are interested in working with
  this example, they can access these sample files at
  \url{http://gking.harvard.edu/cem/imp1.dta}, etc. Once all five are
  downloaded, users can generate the following output. The original data
  file with missingness added is at
  \url{http://gking.harvard.edu/cem/lelonde.dta}}. If we place all of the
imputed datasets in the same directory and open the first, we can run  \cem
with the \texttt{miname} and \texttt{misets} arguments to specify the
root of the imputed datasets' filename and the number of datasets,
respectively.  In our example, this would be:

\begin{verbatim}
. use imp1.dta, clear

. 
. cem age education black nodegree re74, tr(treated) miname(imp) misets(5)

Matching Summary:
-----------------
Number of strata: 235
Number of matched strata: 76

             0    1
      All  425  297
  Matched  312  217
Unmatched  113   80


Multivariate L1 distance: .38286064

Univariate imbalance:

                L1     mean      min      25%      50%      75%      max
      age   .02132  -.07344   .19196        0        1        0       -1
education   .01173   -.0121        1        0        0        0        0
    black   .00207   .00041        0        0        0        0        0
 nodegree   .00461  -.00092        0        0        0        0        0
     re74   .04987  -4.1404  -398.68        0    375.1   -236.7   96.881

\end{verbatim}

Note that \texttt{cem} outputs the data in a single, stacked format so that we
can use the estimation procedure as with a single file above. 

% The output is identical to a normal run of \cem and the output can be
% interpreted similarly. 

% Now we estimate SATT via the usual multiple imputation combining
% formulas (averaging the point estimates and within and between
% variances, as usual; see \citealt{KinHonJos01}), being sure to use the
% \cem weights. This is simple using the \texttt{miest} command by Ken
% Scheve\footnote{\texttt{miest} is available at
%   \url{http://gking.harvard.edu/amelia/amelia1/docs/mi.zip}.}. For example,

% \begin{verbatim}
% . miest imp reg re78 treated [aweight=cem_weights]

% Multiple Imputation Estimates

% Model: regress
% Dependent Variable: re78

% Number of Observations: 529
% ---------------------------------------------------------------
%          |      Coef.   Std. Err.       t          Df     P>|t|
% ---------------------------------------------------------------
%  treated |   1269.2     557.2244      2.278      10902    0.023
%    _cons |   4814.5     355.8442     13.530      22308    0.000
% ---------------------------------------------------------------


% \end{verbatim}


% Note that we have to use \texttt{aweight} instead of \texttt{iweight}
% as above (this is due to compatibility issues). One can use
% \texttt{miest} to implement a number of parametric models with the
% matching weights. In addition, \texttt{clarify}\footnote{Clarify is
%   available at \url{http://gking.harvard.edu/stats.shtml\#clarify}} is
% a useful program for analyzing multiply imputed data:

% \begin{verbatim}
% . estsimp reg re78 treated [iweight=cem_weights], mi(imp1.dta imp2.dta imp3.dta
% >  imp4.dta imp5.dta)

% \oom

% Regress estimates (via multiple imputation)          Nobs = 528

% ---------------------------------------------------------------
%     re78 |      Coef.   Std. Err.       t         d.f.    P>|t|
% ---------+-----------------------------------------------------
%  treated |   1269.205   557.7437      2.276      10943    0.023
%    _cons |    4814.52   356.1777     13.517      22392    0.000
% ---------------------------------------------------------------

% Number of simulations  : 1000
% Names of new variables : b1 b2 b3
% Datasets used for MI   : imp1.dta imp2.dta imp3.dta imp4.dta imp5.dta

% \end{verbatim}


% Note that here we were able to use the \texttt{iweight} command as in our
% earlier analyses. 


\subsection{Blocking in Randomized Experiments} 

CEM can produce strata for a block randomized design for a set of
pre-treatment covariates. As block randomized designs outperform complete
randomization on bias, efficiency, power and robustness, it should be used
whenever possible \citep{ImaKinNal09,ImaKinStu08}. To create a set of
strata for a block randomized design, simply run CEM without passing a
treatment variable. This will assign observations to strata based on their
coarsened values and create a \texttt{cem\_strata} variable, indicating
this assignment. Once this is complete, simply randomly assign treatment
within these strata to complete the block randomized design.

\subsection{Using \texttt{cem} to Improve Other Matching Methods}

Even if you plan to use a different matching method, you can still use the
CEM algorithm to improve that matching solution. An important step before
matching is restricting the data to areas of common empirical support.
This avoids making inference based on extrapolation as such inferences are
known to be extremely model dependent. Traditional matching methods, however,
are not equipped to handle this situation. For example, the propensity
score can be used to find the area of extrapolation only after we know
that the correct propensity score model has been used.  However, the only
way to verify that the correct propensity score model has been specified
is to check whether matching on it produces balance between the treated
and control groups on the relevant covariates.  But balance cannot be
reliably checked until the region of extrapolation has been removed.  To
avoid this type of infinite regress, researchers use entirely different
technologies for the first step, such as kernel density estimation
\citep{HecIchTod97} or dropping control units outside the hyper-rectangle
\citep{IacPor08} or convex hull \citep{KinZen06} of the treated units.

The matching methods currently in Stata all rely on propensity score
methods for restricting the data to common empirical support. For CEM, on
the other hand, this restriction is a natural consequence of the
algorithm. All observations within a stratum containing both a treated and
control unit are by definition inside of the common support. In light of
this, a good use of CEM would be to reduce the data to common support
before applying another matching solution such as
\texttt{psmatch2},\texttt{ nnmatch}, or \texttt{pscore}. This will improve
the quality of the inferences drawn from these methods. Once you have run
\texttt{cem}, all you must do is run the following command to restrict the
data to common support:

\begin{verbatim}
. drop if cem_matched == 0
\end{verbatim}

Alternatively, you can use any of the matching methods with an 
\texttt{if cem\_matched==1} option. This will force the other matching 
methods to only match in the region of common support. As an example using
\texttt{nnmatch}, this would be

\begin{verbatim}
. nnmatch re78 treated age education black nodegree re74  if cem_matched == 1
\end{verbatim}
Of course  you can apply this idea to any matching method in Stata, not
just the ones listed here. 

\section{\texttt{cem} -- Coarsened Exact Matching}

\subsection{Syntax}

.  cem {\it varname1} [({\it cutpoints1})] [{\it varname2}
    [({\it cutpoints2})]] ... [, 
    \underline{tr}eatment({\it varname}) \underline{sh}owbreaks 
    \underline{auto}cuts({\it string}) k2k imbbreaks({\it string}) 
    miname({\it string}) misets(\#)]

\subsection{Description}

\texttt{cem} implements the Coarsened Exact Matching method described in
\citet{IacKinPor12}. The main inputs for \texttt{cem} are the
variables to use (\textit{varname\#}) and the cutpoints that define the
coarsening (\textit{cutpoints\#}). The latter option is set in a
parentheses after the name of the relevant variable. Users can either
specify cutpoints for a variable or allow \texttt{cem} to automatically
coarsen the data based on an automatic coarsening algorithm, chosen by the
user.  To specify a set of cutpoints for a variable, place a numlist in
parentheses after the variable's name. To specify an automatic coarsening,
place a string indicating the binning algorithm to use in parentheses
after the variable's name. To create a certain number of equally
spaced cutpoints including the extreme values, say 10, place \texttt{\#10} in the parentheses (using \texttt{\#0}
will force \texttt{cem} into not coarsening the variable at all). Omitting
the parenthetical statement after the variable name tells \texttt{cem} to
use the default binning algorithm, itself set by \texttt{autocuts}.

Note that character variables are ignored by \texttt{cem}. These variables
will need to be converted into numeric variables using
\texttt{encode}. Coarsening that are not ordinal must be done before
running \texttt{cem} using the \texttt{recode} command, as described
above. 


\subsection{Options}
\hangindent=\parindent\hangafter=1\noindent \texttt{\underline{tr}eatment(varname)} tells \cem which
variable should be used for matching.


\hangindent=\parindent\hangafter=1\noindent\texttt{\underline{sh}owbreaks} will have \cem display the
cutpoints used for each variable on the screen.

\hangindent=\parindent\hangafter=1\noindent\texttt{\underline{auto}cuts({\it string})} sets the default automatic
coarsening algorithm. The default for this is ``sturges''. Any variable
without a \texttt{cutpoint\#} command after its name will use the autocuts
argument.
  
\hangindent=\parindent\hangafter=1\noindent\texttt{k2k} will have \texttt{cem} produce a matching result
that has the same number of treated and control in each matched strata by
randomly dropping observations.
  
\hangindent=\parindent\hangafter=1\noindent\texttt{imbbreaks({\it string})} sets the coarsening method for the
imbalance checks printed after \texttt{cem} runs. This should match
whichever method is used for imbalance checks elsewhere. If either
\texttt{cem} or \texttt{imb}has been run and there is a
\texttt{r(L1\_breaks)} available, this will be the default.  Otherwise, the
default for this is ``scott''

\hangindent=\parindent\hangafter=1\noindent\texttt{miname({\it string})} is the root of the filenames of the
imputed dataset. They should be in the working directory. For example, if
\texttt{miname} were ``imputed'', then the filenames should be
``imputed1.dta'',``imputed2.dta'' and so on.  

\hangindent=\parindent\hangafter=1\noindent\texttt{misets(\#)} is the number of imputed datasets being
used for matching.

\subsection{Output}

The output \cem returns depends on the inclusion of a treatment variable.
If the treatment variable is provided, \cem will match and return the
following three variables in the current dataset:

\hangindent=\parindent\hangafter=1\noindent\texttt{cem\_strata} is the strata number assigned to each
observation by \texttt{cem}.
  
\hangindent=\parindent\hangafter=1\noindent\texttt{cem\_matched} is 1 for a matched observation and 0 for an
unmatched observation.
  
\hangindent=\parindent\hangafter=1\noindent\texttt{cem\_weights} is the weight of the stratum for each
observation.  Strata with unmatched units are give 0 weight and treated
observations are given a weight of 1.
  
\hangindent=\parindent\hangafter=1\noindent\texttt{cem\_treat} when using the multiple imputation features,
\cem outputs this variable, which is the treatment vector used for
matching. \cem applies the same combination rule to treatment as to
strata.

If the options for multiple imputation are used, \cem saves each of these
variables in each of the imputed datasets, allowing for easy use in programs
like \texttt{miest}.

The following are stored as saved results in Stata's memory:

\begin{tabular}{ll}
Scalars &\\
\texttt{r(n\_strata)} & number of strata. \\
\texttt{r(n\_groups)} & number of treatment levels.\\
\texttt{r(n\_mstrata)} & number of strata with matches.\\
\texttt{r(n\_matched)} &  number of matched observations.\\
\texttt{r(L1)} &  multivariate imbalance measure. \\
Matrices&\\
\texttt{r(match\_table)} & table of treatment vs matched.\\
\texttt{r(groups)}&  tabulation of treatment variable.\\
\texttt{r(imbal)} &  univariate imbalance measures.\\
Macros&\\
\texttt{r(varlist)} & covariate variables used.\\
\texttt{r(treatment)} & treatment variable.\\
\texttt{r(cem\_call)} & call to cem. \\
\texttt{r(L1\_breaks)} &break method used for L1 distance
\end{tabular}

If the treatment variable is omitted (e.g. for blocking), then the only outputs
are \texttt{cem\_strata}, \texttt{r(n\_strata)}, \texttt{r(varlist)}, and
\texttt{r(cem\_call)}.

\section{\texttt{imb} - Imbalance Measures for CEM}

\subsection{Syntax}
imb {\it varlist} [if] [in] [, \underline{tr}eatment({\it varname})
  breaks({\it string}) miname({\it string}) misets({\it string}) \underbar{use}weights]
\subsection{Description}

\texttt{imb} returns a number of measures of imbalance in covariates
between treatment and control groups. A multivariate L1 distance,
univariate L1 distrances, difference in means and empirical quantiles
difference are reported. The L1 measures are computed by coarsening the
data according to \texttt{breaks} and comparing across the multivariate
histogram. See \cite{IacKinPor12} for more details on this measure.
 
\subsection{Arguments}
\hangindent=\parindent\hangafter=1\noindent\texttt{treatment(varname)} sets the treatment variable used for
  the imbalance checks.
  
  \hangindent=\parindent\hangafter=1\noindent\texttt{breaks(string)} sets the default automatic coarsening
  algorithm. If either \texttt{cem} or \texttt{imb} has been run and
  there is a \texttt{r(L1\_breaks)} available, this will be the default.
  Otherwise, the default for this is ``scott''. It is not incredibly
  important which method is used here as long as it is consistent.
  
\hangindent=\parindent\hangafter=1\noindent\texttt{miname(string)} is the root of the filenames of the imputed
  dataset. They should be in the working directory. For example, if
  \texttt{miname} were ``imputed'', then the filenames should be
  ``imputed1.dta'',``imputed2.dta'' and so on.
  
\hangindent=\parindent\hangafter=1\noindent\texttt{misets(integer)} is the number of imputed datasets being
  used for matching.

\hangindent=\parindent\hangafter=1\noindent\texttt{useweights} makes \texttt{imb}use the weights from the output of \texttt{cem}. This is useful for checking balance after running \texttt{cem}.

\subsection{Saved Results}
\begin{tabular}{ll}
Scalars&\\
\texttt{r(L1)} & multivariate imbalance measure\\
Matrices &\\
\texttt{r(imbal)} & matrix of univariate imbalance measures\\
Macros &\\
\texttt{r(L1\_breaks)}& break method used for L1 distance

\end{tabular}

\bibliographystyle{apsr} 
\bibsep=0in 
\bibliography{gk,gkpubs}

\end{document}

%%% Local Variables: 
%%% mode: latex
%%% TeX-master: t
%%% End: 
